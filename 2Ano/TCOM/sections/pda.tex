\documentclass[../resumosTCOM.tex]{subfiles}

\newenvironment{conditions}
  {\par\vspace{\abovedisplayskip}\noindent\begin{tabular}{>{$}l<{$} @{${}={}$} l}}
  {\end{tabular}\par\vspace{\belowdisplayskip}}

\begin{document} 

O PDA (pushdown automata) é um $\epsilon$-NFA com uma stack de símbolos
\begin{itemize}
    \item Adiciona a possibilidade de memorizar uma quantidade infinita de informação.
    \item O PDA só tem acesso ao topo da stack (LIFO).
    \item Como funciona?
    \begin{itemize}
        \item A unidade de controlo lê e consome os símbolos do input.
        \item Transição para um novo estado baseado no estado atual, símbolo do input e símbolo no topo da stack.
        \item Transições espontâneas com $\epsilon$.
        \item Topo da stack substituído por símbolos.
    \end{itemize}
\end{itemize}

\paragraph{}

PDA P = (\(Q, \sum, \Gamma, \delta, q_0, Z_0, F\))
\begin{itemize}
    \item Q: conjunto dos estados.
    \item $\sum$: alfabeto.
    \item $\Gamma$: alfabeto finito da stack.
    \item $\delta$: função de transição.
    \item $q_0$: estado inicial.
    \item $Z_0$: símbolo incial da stack.
    \item F: conjunto dos estados finais (se for um PDA de estado final).
\end{itemize}

\paragraph{}

Transições representadas cmo a, X/$\alpha$
\begin{itemize}
    \item a: símbolo do input consumido.
    \item X: topo da stack (que será retirado - pop).
    \item $\alpha$: string a colocar na stack (push) - valor mais à esquerda ficará no topo da stack.
\end{itemize}

\paragraph{}

Descrição instantânea (q, $\omega$, $\gamma$) - (ID):
\begin{itemize}
    \item q: estado.
    \item $\omega$: input restante.
    \item $\gamma$: conteúdo da stack.
\end{itemize}

\paragraph{}

As CFLs definidas por uma CFG são as linguagens aceites por um PDA por empty stack e também aceites por um PDA por final state.

\paragraph{}

Conversão de PDAs em CFGs:
\begin{itemize}
    \item O evento principal do processamento de um PDA é retirar um símbolo da stack enquanto se consome o input.
    \item Adicionam-se variáveis à gramática para cada:
    \begin{itemize}
        \item eliminação de um símbolo da stack X.
        \item transiçã de p para q eliminando X, representado pelo símbolo composto [pXq]
    \end{itemize}
\end{itemize}

\paragraph{}

Determinismo; \(|\delta(q, s, t)| + |\delta(q, \epsilon, t)| \leq 1\)
\begin{itemize}
    \item q: estado.
    \item s: input.
    \item t: topo da stack.
\end{itemize}

\end{document}

