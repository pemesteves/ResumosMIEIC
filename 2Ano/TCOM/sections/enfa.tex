\documentclass[../resumosTCOM.tex]{subfiles}

\newenvironment{conditions}
  {\par\vspace{\abovedisplayskip}\noindent\begin{tabular}{>{$}l<{$} @{${}={}$} l}}
  {\end{tabular}\par\vspace{\belowdisplayskip}}

\begin{document} 

$\epsilon$-NFA: NFA com transições \(\epsilon\)

\paragraph{}

$\epsilon$-NFA E = (\(Q, \sum, \delta, q_0, F\))
\begin{itemize}
    \item Maior diferença é que a função de transição \(\delta\) lida com \(\epsilon\).
    \begin{itemize}
        \item \(\delta(q, a)\): estado \(q in Q\) e \(a in \sum \bigcup \{\epsilon\}\)
    \end{itemize}
    \item \(\epsilon\) representa transições espontâneas.
    \item Para saber quais os estados que conseguimos alcançar a partir de um estado q com \(\epsilon\), calculamos o $\epsilon$-close(q). Exemplos:
    \begin{itemize}
        \item $\epsilon$-close\((q_0) = \{q_0, q_1\}\)
        \item $\epsilon$-close\((q_3) = \{q_3, q_5\}\)
    \end{itemize}
\end{itemize}

\paragraph{}

Para um dado $\epsilon$-NFA existe sempre um DFA equivalente.

\end{document}

