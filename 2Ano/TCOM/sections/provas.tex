\documentclass[../resumosTCOM.tex]{subfiles}

\newenvironment{conditions}
  {\par\vspace{\abovedisplayskip}\noindent\begin{tabular}{>{$}l<{$} @{${}={}$} l}}
  {\end{tabular}\par\vspace{\belowdisplayskip}}

\begin{document} 

Métodos de Prova [if H then C (H \(\rightarrow\) C)]
\begin{itemize}
    \item Por Contradição [H and not C implies falsehood]
    \item Por Contra-exemplo: mostrar um exemplo que prova que a proposição é falsa
    \item Por Contra-positivo [if not C then not H]: provar um é provar o outro
    \item Por Indução (provar S(n)):
    \begin{itemize}
        \item Caso base (Provar S(i), sendo i o 1º valor, normalmente 0 ou 1)
        \item Passo indutivo (Assumindo a hipótese, prova-se S(n+1))
        \item Sendo n um número geral, a propriedade aplica-se a todos os n
    \end{itemize}
\end{itemize}

\end{document}

