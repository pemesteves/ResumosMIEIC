\documentclass[../resumosTCOM.tex]{subfiles}

\newenvironment{conditions}
  {\par\vspace{\abovedisplayskip}\noindent\begin{tabular}{>{$}l<{$} @{${}={}$} l}}
  {\end{tabular}\par\vspace{\belowdisplayskip}}

\begin{document} 

Controlador de estados finitos (número de estados finitos).

\paragraph{}

Fita com comprimento infinito que consiste em células (cada célula pode conter um símbolo).

\paragraph{}

Input: string finita, que consiste em símbolos do alfabeto do input, colocada no início da fita (todas as outras células são marcadas com B).

\paragraph{}

Símbolos na fita: alfabeto do input + blank (B) + outros símbolos necessários.

\paragraph{}

TM T = (\(Q, \sum, \Gamma, \delta, q_0, B, F\))
\begin{itemize}
    \item Q: estados da TM.
    \item $\sum$: alfabeto do input.
    \item $\Gamma$: alfabeto da fita.
    \item $\delta$: função de transição.
    \item $q_0$: estado inicial.
    \item B: blank.
    \item F: conjunto dos estados finais.
\end{itemize}

\paragraph{}

Transições entre estados ($0/X \rightarrow$):
\begin{itemize}
    \item 0: símbolodo input.
    \item X: símbolo colocado na fita.
    \item $\rightarrow$: direção da leitura.
\end{itemize}

\paragraph{}

$\delta(q, X) = (p, Y, D)$
\begin{itemize}
    \item q e p: estados inicial e final da transição.
    \item X e Y: símbolos da fita (o que está na fita e o que fica na fita, respetivamente).
    \item D: direção da leitura.
\end{itemize}

\end{document}

