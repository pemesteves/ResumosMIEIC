\documentclass[../resumosTCOM.tex]{subfiles}

\newenvironment{conditions}
  {\par\vspace{\abovedisplayskip}\noindent\begin{tabular}{>{$}l<{$} @{${}={}$} l}}
  {\end{tabular}\par\vspace{\belowdisplayskip}}

\begin{document} 

Notação que permite especificar linguagens não regulares (algumas).

\paragraph{}

CFG G = (V, T, P, S) -> tuplo das CFGs
\begin{itemize}
    \item V: variáveis da linguagem
    \item T: terminais, ou seja, símbolos usados nas strings (alfabeto)
    \item P: produções ou regras da linguagem
    \item S: símbolo (variável) inicial
\end{itemize}

\paragraph{}

Derivação: a partir de uma string, aplicam-se as "regras" da linguagem
\begin{itemize}
    \item Leftmost ($\xRightarrow[lm]{*}$): substituem-se primeiro as variáveis à esquerda.
    \item Rightmost ($\xRightarrow[rm]{*}$): substituem-se primeiro as variáveis à direita.
\end{itemize}

\paragraph{}

Syntax trees (árvores de análise)
\begin{itemize}
    \item Estrutura de dados mais usada para representar o programa de input num compilador
    \item Cada nó representa uma variável da gramática, um terminal ou $\epsilon$
\end{itemize}

\paragraph{}

Eliminação de ambiguidade: podem-se adicionar novas variáveis para distinguir níveis de prioridade e regras de associação.

\paragraph{}

Uma CFL é ambígua se todas as gramáticas para L são ambíguas.

\end{document}

