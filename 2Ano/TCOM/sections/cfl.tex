\documentclass[../resumosTCOM.tex]{subfiles}

\newenvironment{conditions}
  {\par\vspace{\abovedisplayskip}\noindent\begin{tabular}{>{$}l<{$} @{${}={}$} l}}
  {\end{tabular}\par\vspace{\belowdisplayskip}}

\begin{document} 

\textbf{Chomsky Normal Form (CNF)}: simplificação de CFGs
\begin{itemize}
    \item Eliminação de símbolos não usados.
    \item Eliminação de produções $\epsilon$.
    \item Eliminação de produções unitárias.
    \item Variáveis dos terminais quando estão juntos são isoladas.
    \item Substitui-se os terminais por estas variáveis.
    \item Substituem-se corpos longos.
\end{itemize}

\paragraph{}

Verificar se uma string pertence a uma CFL
\begin{itemize}
    \item \textbf{Cocke-Younger-Kasami (CYK) Algorithm}
    \begin{itemize}
        \item $x_{ij}$: representa o conjunto de variáveis que produz a string i-j.
        \item $O(N^3)$ usando programação dinâmica, preenche-se a tabela ($a_1a_2a_3a_4$ é a string de input):
        
        \begin{table}[H]
            \centering
            \begin{tabular}{llll}
            \hline
            \multicolumn{1}{|l|}{$x_{14}$} & \multicolumn{1}{l|}{}         & \multicolumn{1}{l|}{}         & \multicolumn{1}{l|}{}         \\ \hline
            \multicolumn{1}{|l|}{$x_{13}$} & \multicolumn{1}{l|}{$x_{24}$} & \multicolumn{1}{l|}{}         & \multicolumn{1}{l|}{}         \\ \hline
            \multicolumn{1}{|l|}{$x_{12}$} & \multicolumn{1}{l|}{$x_{23}$} & \multicolumn{1}{l|}{$x_{34}$} & \multicolumn{1}{l|}{}         \\ \hline
            \multicolumn{1}{|l|}{$x_{11}$} & \multicolumn{1}{l|}{$x_{22}$} & \multicolumn{1}{l|}{$x_{33}$} & \multicolumn{1}{l|}{$x_{44}$} \\ \hline
            $a_1$                          & $a_2$                         & $a_3$                         & $a_4$                        
            \end{tabular}
        \end{table}
        \[X_{13} = X_{12}X_{22} \cup X_{11}X_{23}\]
    \end{itemize}
\end{itemize}

\paragraph{}

\textbf{Pumping Lemma para CFLs}
\begin{itemize}
    \item Assume-se que L é uma CFL.
    \item Então, existe uma constante n com a qual, para todo o z em L com \(|z| \geq n\), conseguimos escrever z = uvwxy:
    \begin{itemize}
        \item \(|vwx| \leq n\)
        \item \(vx \neq \epsilon\) (pelo menos uma, v ou x, não pode ser a string vazia)
        \item Para todo \(i \geq 0\), \(uv^iwx^iy\)
    \end{itemize}
\end{itemize}

\paragraph{}

Interseção de uma LR com uma CFL resulta numa CFL.

\end{document}

