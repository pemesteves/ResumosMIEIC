\documentclass[../resumosLPOO.tex]{subfiles}

\newenvironment{conditions}
  {\par\vspace{\abovedisplayskip}\noindent\begin{tabular}{>{$}l<{$} @{${}={}$} l}}
  {\end{tabular}\par\vspace{\belowdisplayskip}}

\begin{document} 

Diagrama de classes: diagrama estrutural.

\paragraph{}

Mostram as classes do sistema, as suas operações, atributos e relações.

\paragraph{}

Atributos (secção do meio): \lstinline{nome:tipo_do_atributo=valor_por_defeito} (valor por defeito é opcional)

\paragraph{}

Operações (secção de baixo): \lstinline{nome(list_parâmetros):tipo_retornado}
\begin{itemize}
    \item Cada parâmetro da lista: \lstinline{nome:tipo}
    \item Os parâmetros podem ter uma marca "in" ou "out" indicando se é um input ou output
\end{itemize}

\paragraph{}

Herança: aponta da subclasse para a superclass

\begin{figure}[H]
    \centering
    %\resizebox{\textwidth}{!}{
    \begin{tikzpicture}

    \umlclass[x=0, y=0]{A}
    {}
    {}


    \umlclass[x=0, y=-2]{B}
    {}
    {}

    \umlinherit[]{B}{A}

    \end{tikzpicture}
    %}
    \caption{Exemplo Herança\label{fig:inheritance}}
\end{figure}

\paragraph{}

Classes e métodos abstratos são representados em itálico.

\paragraph{}

Associações bidirecionais: cada lado tem uma multiplicidade.

\begin{figure}[H]
    \centering
    %\resizebox{\textwidth}{!}{
    \begin{tikzpicture}

    \umlclass[x=-2, y=0]{A}
    {}
    {}


    \umlclass[x=2, y=0]{B}
    {}
    {}

    \umlassoc[geometry=-|-, mult1=*, pos1=0.2, mult2=0, pos2=2.8]{A}{B}

    \end{tikzpicture}
    %}
    \caption{Exemplo Associação Bidirecional\label{fig:inheritance}}
\end{figure}

\paragraph{}

Associações unidirecionais: só uma classe sabe que a relação existe - aquela de onde parte a seta.

\begin{figure}[H]
    \centering
    %\resizebox{\textwidth}{!}{
    \begin{tikzpicture}

    \umlclass[x=-2, y=0]{A}
    {}
    {}


    \umlclass[x=2, y=0]{B}
    {}
    {}

    \umluniassoc[geometry=-|-, mult1=*, pos1=0.2, mult2=0, pos2=2.8]{A}{B}

    \end{tikzpicture}
    %}
    \caption{Exemplo Associação Unidirecional\label{fig:inheritance}}
\end{figure}

\paragraph{}

Classe de associação: inclui informação sobre a relação (ligada à associação com ---)

\paragraph{}

Interfaces: classe com <<interface>> antes do nome. 

Implementação de uma interface:
\begin{figure}[H]
    \centering
    %\resizebox{\textwidth}{!}{
    \begin{tikzpicture}

    \umlinterface[x=0, y=2.5]{A}
    {}
    {}


    \umlclass[x0, y=0]{B}
    {}
    {}

    \umlimpl[]{B}{A}

    \end{tikzpicture}
    %}
    \caption{Exemplo Implementação Interface\label{fig:inheritance}}
\end{figure}

\paragraph{}

Agregação: modela um todo para as partes.
\begin{figure}[H]
    \centering
    %\resizebox{\textwidth}{!}{
    \begin{tikzpicture}

    \umlclass[x=0, y=2.5]{A}
    {}
    {}


    \umlclass[x0, y=0]{B}
    {}
    {}

    \umlaggreg[]{A}{B}

    \end{tikzpicture}
    %}
    \caption{Exemplo Agregação\label{fig:inheritance}}
\end{figure}

\paragraph{}

Composição: forma mais forte de agregação. O todo só existe com as suas partes.
\begin{figure}[H]
    \centering
    %\resizebox{\textwidth}{!}{
    \begin{tikzpicture}

    \umlclass[x=0, y=2.5]{A}
    {}
    {}


    \umlclass[x0, y=0]{B}
    {}
    {}

    \umlcompo[]{A}{B}

    \end{tikzpicture}
    %}
    \caption{Exemplo Composição\label{fig:inheritance}}
\end{figure}

\end{document}

