\documentclass[../resumosLPOO.tex]{subfiles}

\newenvironment{conditions}
  {\par\vspace{\abovedisplayskip}\noindent\begin{tabular}{>{$}l<{$} @{${}={}$} l}}
  {\end{tabular}\par\vspace{\belowdisplayskip}}

\begin{document} 

Genéricos providenciam Type-safety.

\paragraph{}

Type Variable: identificador não qualificado (ex: <T>).
\begin{itemize}
    \item Nomes:
    \begin{itemize}
        \item \textbf{E} - Elemento
        \item \textbf{K} - Chave (Key)
        \item \textbf{N} - Número
        \item \textbf{T} - Tipo (S, U, V.. -> 2º, 3º e 4º tipos)
        \item \textbf{V} - Valor
    \end{itemize}
\end{itemize}

\paragraph{}

Métodos genéricos: \lstinline{public <T> void arrayToList(T[] a, List<T> l)}

\paragraph{}

Classes genéricas: \lstinline{pubic class Box<T>}

É possível criar classes que estendem tipos genéricos.

\paragraph{}

Variância:
\begin{itemize}
    \item \textbf{Covariância}: preserva a ordem dos tipos (aceita subtipos) -> tipos de Java.
    \item \textbf{Contravariância}: troca a ordem dos tipos (aceita supertipos).
    \item \textbf{Invariância}: só aceita o tipo específico -> Java Generics.
\end{itemize}

\paragraph{}

Wildcards: 
\begin{itemize}
    \item em vez de <T> é possível utilizar <?> (tipo desconhecido).
    \item possível utilizar <? extends Animal> (variante) ou <? super Animal> (contravariante).
\end{itemize}

\end{document}

