\documentclass[../resumosLPOO.tex]{subfiles}

\newenvironment{conditions}
  {\par\vspace{\abovedisplayskip}\noindent\begin{tabular}{>{$}l<{$} @{${}={}$} l}}
  {\end{tabular}\par\vspace{\belowdisplayskip}}

\begin{document} 

Diagrama de sequência: diagrama comportamental.

\paragraph{}

Mostra a interação entre os objetos numa ordem sequencial.

\paragraph{}

Lifeline: elemento com nome que representa um indivíduo na interação.

\paragraph{}

Linha vertical representa o tempo (de cima para baixo).

\paragraph{}

Actor: sistema ou pessoa que está fora do escopo do sistema.

\begin{center}
    \begin{tikzpicture}
        \begin{umlseqdiag}
            \umlactor[class=A]{a}
            \umlcreatecall[class=E, x=8]{a}{e}
            \begin{umlcall}[op=op(), name=test, type=synchron, return=6, dt=7, fill=red!10]{a}{e}
            \end{umlcall}
        \end{umlseqdiag}
    \end{tikzpicture}
\end{center}

\paragraph{}

Mensagens: seta a partir do objeto que envia para o que recebe. Nome do método por cima da seta.
Retorno é opcional e representa-se com uma seta a tracejado.

\begin{center}
    \begin{tikzpicture}
        \begin{umlseqdiag}
            \umlactor[class=A]{a}
        \end{umlseqdiag}
    \end{tikzpicture}
\end{center}




\end{document}

