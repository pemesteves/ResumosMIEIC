\documentclass[../resumosLPOO.tex]{subfiles}

\newenvironment{conditions}
  {\par\vspace{\abovedisplayskip}\noindent\begin{tabular}{>{$}l<{$} @{${}={}$} l}}
  {\end{tabular}\par\vspace{\belowdisplayskip}}

\begin{document} 

Um code smell nem sempre indica um problema, não é um problema.

Alguns code smells:
\begin{itemize}
    \item \textbf{Long Method}: método com muitas linhas.
    \item \textbf{Large Class}: classe contém muitos atributos/métodos/linhas de código.
    \item \textbf{Long Parameter List}: mais de 3 ou 4 parâmetros por método.
    \item \textbf{Data Clumps}: partes diferentes do código contêm o mesmo grupo de variáveis.
    \item \textbf{Switch Statements}: operações if/switch complexas.
    \item \textbf{Refused Biquest}: subclasse só usa alguns métodos/atributos herdados da superclasse.
    \item \textbf{Alternative Classes with Different Interfaces}: duas classes têm funções idênticas, mas nomes de métodos diferentes.
    \item \textbf{Divergent Change}: mudar muitos métodos não relacionados quando se faz alterações numa classe.
    \item \textbf{Lazy Class}: classes que não fazem muita coisa.
    \item \textbf{Data Class}: classe que apenas contém atributos e métodos para lhes aceder.
    \item \textbf{Feature Envy}: método que acessa mais aos dados de outro objeto que aos seus dados.
\end{itemize}

Técnicas de Refactoring:
\begin{itemize}
    \item Extract Method / Class
    \item Inline Method / Class
    \item Extract Variable
    \item Split Temporary Variable
    \item Decompose Conditional
    \item Consolidate Duplicate Conditional Fragments
    \item Replace Nested Conditional with Guard Clauses
    \item Introduce Null Object
\end{itemize}

\end{document}

