\documentclass[../resumosLPOO.tex]{subfiles}

\newenvironment{conditions}
  {\par\vspace{\abovedisplayskip}\noindent\begin{tabular}{>{$}l<{$} @{${}={}$} l}}
  {\end{tabular}\par\vspace{\belowdisplayskip}}

\begin{document} 

Testes black-box: estrutura interna é desconhecida ou não considerada.

\paragraph{}

Testes white-box: o design é baseado na estrutura interna para que o número máximo de "caminhos" do código sejam testados.

\paragraph{}

\textbf{Unit Testing}:
\begin{itemize}
    \item testar unidades individuais de software
    \item código necessita de ser modular, o que o faz ser reutilizável
\end{itemize}

\paragraph{}

\textbf{Princípios FIRST}:
\begin{itemize}
    \item \textbf{F}ast: devem ser rápidos.
    \item \textbf{I}solated/\textbf{I}ndependent: só se testa uma unidade de cada vez. Ordem não interessa.
    \item \textbf{R}epeatable: não devem depender do ambiente (time, random values, ...).
    \item \textbf{S}elf-validating: não é necessário verificar à mão.
    \item \textbf{T}horough/\textbf{T}imely: devem cobrir todos os casos de uso.
\end{itemize}

\paragraph{}

Três A's nos quais os testes devem ser divididos:
\begin{itemize}
    \item \textbf{A}rrange: inicialização.
    \item \textbf{A}ct: método a testar é invocado.
    \item \textbf{A}ssert: é usado um assert para testar o resultado.
\end{itemize}

\paragraph{}

\textbf{Stubs}: providenciam respostas para as chamadas realizadas.

\paragraph{}

\textbf{Mocks}: pré-programados com expectativas que geram uma especificação das chamadas que se espera receber.

\paragraph{}

\textbf{State Testing}: testar o estado após invocação do método.

\paragraph{}

\textbf{Behaviour Testing}: testar o comportamento do método.

\end{document}

