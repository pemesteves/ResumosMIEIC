\documentclass[../resumosMNUM.tex]{subfiles}

\newenvironment{conditions}
  {\par\vspace{\abovedisplayskip}\noindent\begin{tabular}{>{$}l<{$} @{${}={}$} l}}
  {\end{tabular}\par\vspace{\belowdisplayskip}}

\begin{document} 

\subsection{Método da Bisseção}

A partir de um intervalo, calcula-se o seu ponto médio. Se \(f(\frac{x_1+x_2}{2})\) for nulo, encountrou-se a raíz; se não, reduz-se o intervalo.

C++:
\begin{lstlisting}
    for(int n= 0; n < ...; n++) {
        m = (a + b) / 2;
        if(f(a) * f(m) < 0)
            b = m;
        else if(f(a) * f(m) > 0)
            a = m;
        else
            break;
    }
\end{lstlisting}

Critérios de Paragem:
\begin{enumerate}
    \item \(|x_1 - x_2| \leq \epsilon\) - critério de precisão absoluta
    \item \(\frac{|x_1 - x_2|}{x_1} \leq \epsilon\) ou \(\frac{|x_1 - x_2|}{x_2} \leq \epsilon\) - critério de precisão relativa
    \item \(|f(x_1) - f(x_2)| \leq \epsilon\) - critério de anulação da função
    \item \(n = N\) - critério do número de iterações
    \begin{itemize}
        \item \(n \leq (número de bits da mantissa) + \log_2(b - a)\) -> a e b são os extremos do intervalo 
        \item ou \(n \leq 3.3 * (número de dígitos da mantissa) + \log_{10}(b - a)\)
    \end{itemize}
\end{enumerate}

\subsection{Método da Corda ou Falsa Posição}

Calcula-se a nova posição traçando-se uma corda entre os dois pontos.

\[m = \frac{a * f(b) - b * f(a)}{f(b) - f(a)}\]

C++:
\begin{lstlisting}
    for(int n = 0; n < ...; n++) {
        w = (a * f(b) - b * f(a))/(f(b) - f(a));
        if(f(a) * f(w) < 0)
            b = w;
        else if(f(a) * f(w) > 0)
            a = w;
        else
            break;
    }
\end{lstlisting}

\subsection{Método da Tangente ou Newton}

Parte apenas de um valor plausível, substituindo este valor pelo zero da tangente neste ponto

\[x_{k + 1} = x_k - \frac{f(x_k)}{f'(x_k)}\]

\begin{itemize}
    \item Implica conhecimento prévio da derivada e \(f'(x_k) \neq 0\)
\end{itemize}

\subsection{Método de Picard-Peano}

Transforma-se uma equação \(f(x) = 0\) em \(x = g(x)\), tal que g(x) não seja divergente. Através da equação \(x_n = g(x_{n - 1})\) calculam-se pontos sucessivos até se chegar perto do 0

\begin{itemize}
    \item Condição de convergência: \(g'(x) < 1\)
    \item Para sistemas de equação:  \begin{cases} $x = g_1(x, y)$ \\ $y = g_2(x, y)$ \end{cases}
    \newline
    Todas as derivadas parcias têm que convergir: \(\frac{\delta g_1}{\delta x}\), \(\frac{\delta g_1}{\delta y}\), \(\frac{\delta g_2}{\delta x}\), \(\frac{\delta g_2}{\delta y}\)

\end{itemize}

\end{document}

