\documentclass[../resumosMNUM.tex]{subfiles}

\newenvironment{conditions}
  {\par\vspace{\abovedisplayskip}\noindent\begin{tabular}{>{$}l<{$} @{${}={}$} l}}
  {\end{tabular}\par\vspace{\belowdisplayskip}}

\begin{document} 

\subsection{Método de Gauss}

\begin{lstlisting}
    m: matrix([x1, y1, z1, b1], [x2, y2, z2, b2], [x3, y3, z3, b3])$
    
    for i:1 thru 3 do (
        m: rowop(m, i, i, 1 - 1/m[i][i]),
        for j:1 thru 3 do (
            if(i#j) then m: rowop(m, j, i, m[j][i]))) $
\end{lstlisting}

Este método pressupõe que m[i][i] não é nulo.


\subsection{Runge-Kutta 4}

\begin{lstlisting}
    rk(f', y, 1, [x, 0, 4, 1]);
\end{lstlisting}

f': derivada da função

y: variável

1: valor inicial de y

[x, 0, 4, 1]: [x, x inicial, x final, h]

\paragraph{}

Sistemas: 

\begin{lstlisting}
    rk([x', y'], [x, y], [-1.25, 0.75], [t, 0, 4, h]);
\end{lstlisting}

\subsection{Khaletsky}

\begin{lstlisting}
    A: matrix([1, 2, 3], [4, 5, 6], [7, 8, 9]);
    b: [10, 11, 12];
    [P, L, U]: get_lu_factors(lu_factor(A));
    Y: invert(L).b;
    X: invert(U).Y;
\end{lstlisting}

\subsection{Hessiana}

\qquad hessian(função, [lista de variáveis]);

\end{document}

