\documentclass[../resumosMNUM.tex]{subfiles}

\newenvironment{conditions}
  {\par\vspace{\abovedisplayskip}\noindent\begin{tabular}{>{$}l<{$} @{${}={}$} l}}
  {\end{tabular}\par\vspace{\belowdisplayskip}}

\begin{document} 

\subsection{Regra dos Trapézios - 2ª ordem}

Substitui-se, em cada intervalo, o arco da curva pela sua corda, calculando, em seguida, a área sob a poligonal assim definida.

\[ \int_{x_0}^{x_n} y.\delta x = \frac{h}{2} . [y_0 + 2y_1 + ... + 2y_{n-1} + y_n]\]

\subsection{Controlo do erro}

Quociente de Convergência: \(\frac{S' - S}{S'' - S'} \approx 2^{ordem do método}\)

Erro: \(\epsilon '' = \frac{S'' - S'}{2^{ordem} - 1}\)

\subsection{Regra de Simpson - 4ª ordem}

Em vez de substituir a curva por cordas, substitui-a pelas parábolas definidas por cada trio de pontos.

\[ \int_{x_0}^{x_{2n}} y.\delta x = \frac{h}{3} . [y_0 + 4y_1 + 2y_2 + 4y_3 + ... + 2y_{2n-2} + 4y_{2n-1} + y_{2n}]\]

\paragraph{}

Fórmula de Simpson - Cubatura:

\[h_x = \frac{A - a}{2}\]

\[h_y = \frac{B - b}{2}\]

\[ \int \int f(x, y) \delta x \delta y = \frac{h_x . h_y}{9} . [\sum_0 + 4\sum_1 + 16\sum_2]\]

$\sum_0$: Valores de f nos vértices 

$\sum_1$: Valores de f nos pontos médios dos lados 

$\sum_0$: Valores de f no centro

\end{document}

