\documentclass[../resumosPLOG.tex]{subfiles}

\newenvironment{conditions}
  {\par\vspace{\abovedisplayskip}\noindent\begin{tabular}{>{$}l<{$} @{${}={}$} l}}
  {\end{tabular}\par\vspace{\belowdisplayskip}}

\begin{document} 

findall(Term, Goal, Bag): unifica Bag com a lista das instâncias de Term para as quais Goal é satisfeito.
\begin{itemize}
    \item Todos os X para os quais \lstinline{call(Goal), X=Term?} é satisfeito.
    \item Term e Goal tipicamente partilham variáveis.
\end{itemize}

\paragraph{}

bagof(Term, Goal, Bag): idêntico ao findall, mas são encontradas soluções alternativas para as variáveis em Goal.

\paragraph{}

setof(Term, Goal, Bag): soluções ordenadas, sem duplicados (conjunto).

\paragraph{}

Meta-Interpretador: interpretador de uma linguagem escrito na própria linguagem.
\begin{itemize}
    \item Em Prolog, é fácil construí-los porque não há distinção entre programa e dados.
    \item Interesse em desenvolvê-los:
    \begin{itemize}
        \item Implementar diferentes estratégias de pesquisa da solução.
        \item Incluir capacidade de explicação.
        \item Incluir facilidades acrescidas de traçagem, teste e debugging.
    \end{itemize}
\end{itemize}

\paragraph{}

Definição de operadores:
\begin{itemize}
    \item Nome (um átomo), tipo (classe e associatividade) e prioridade (inteiro entre 1 e 1200).
    \item :- op(Prioridade, Tipo, Nome).
    \item Tipos de Operadores: fx, fy, xfx, xfy, yf, xf (y tem prioridade; x e y indicam o lado da associação).
\end{itemize}

\end{document}

