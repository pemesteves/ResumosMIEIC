\documentclass[../resumosPLOG.tex]{subfiles}

\newenvironment{conditions}
  {\par\vspace{\abovedisplayskip}\noindent\begin{tabular}{>{$}l<{$} @{${}={}$} l}}
  {\end{tabular}\par\vspace{\belowdisplayskip}}

\begin{document} 

Domínio das variáveis:
\begin{itemize}
    \item Uma variável pode ter o seu domínio declarado usando in/2 e um intervalo (Constant Range).
    \item Pode ainda ser usado in\_set/2. O seu segundo argumento é um Finite Domain Set, que pode ser obtido a partir de uma lista com o predicado list\_to\_fdset(+List, -FD\_Set).
    \item Para declara o mesmo domínio a uma lista de variáveis pode ser usado o predicado domain(+List\_of\_Variables, +Min, +Max).
\end{itemize}

\end{document}

