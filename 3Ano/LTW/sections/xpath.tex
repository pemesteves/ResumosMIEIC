\documentclass[../resumosLTW.tex]{subfiles}

\newenvironment{conditions}
  {\par\vspace{\abovedisplayskip}\noindent\begin{tabular}{>{$}l<{$} @{${}={}$} l}}
  {\end{tabular}\par\vspace{\belowdisplayskip}}

\begin{document} 

Linguagem para endereçar partes de um documento XML.

\paragraph{}

Modela um documento XML como uma árvore de nós.

\paragraph{}

Há diferentes tipos de nós: elementos, atributos e texto.

\paragraph{}

Os tipos de nós são usados para representar o documento como uma árvore:
\begin{itemize}
    \item Um document node é a raíz da árvore.
    \item Cada nó element representa uma tag XML.
    \item Atributos de um elemento são representados por attribute nodes.
    \item Texto dentro de um elemento torna-se um text node.
    \item Comentários são representados como comment nodes.
    \item Construções XML \lstinline{<?...?>} tornam-se processing instructions nodes.
\end{itemize}

\paragraph{}

Tipos de dados usados por expressões XPath:
\begin{itemize}
    \item node-set: conjunto de 0 ou mais nós.
    \item boolean: valor true ou false.
    \item number: números em XPath são representados em vírgula flutuante.
    \item string: cadeia de caracteres.
\end{itemize}

\paragraph{}

Location Path:
\begin{itemize}
    \item Seleciona um conjunto de elementos relativo ao content node.
    \item Se for precedido por /, torna-se um caminho absoluto e o context node é a raíz do documento.
    \item Um location path é composto por location steps, separados por /, cada um com 3 partes:
    \begin{itemize}
        \item um eixo (axis).
        \item um nó de teste (test node).
        \item zero ou mais predicados.
    \end{itemize}
\end{itemize}

\paragraph{}

Eixo (axis): especifica a relação de árvore entre os nós selecionados pelo location step e o context node.

\paragraph{}

Cada eixo tem um tipo principal de nó. 
Se um eixo pode conter elementos, o tipo de nó principal é element; caso contrário, é o tipo de nós que o eixo contém.
\begin{itemize}
    \item Para o eixo attribute, o tipo de nó principal é attribute.
    \item Para o eixo namespace, o tipo de nó principal é namespace.
    \item Para outros eixos, o tipo de nó principal é element.
\end{itemize}

\paragraph{}

Um node test, que é um QName, é verdadeiro se e só se o tipo de nó é o tipo de nó principal e tem um nome igual ao nome especificado pelo QName.
O node test * é verdadeiro para qualquer text node.
O node test text() é verdadeiro para qualquer text node.
O node test comment() é verdadeiro para qualquer comment node.
O node test processing-instruction() é verdadeiro para qualquer processing instruction node.
Um node test node() é verdadeiro para qualquer nó.

\paragraph{}

Predicados encontram-se entre [] e selecionam nós de conjunto. 
Um location step tem 0 ou mais predicados.

\paragraph{}

Abreviações:
\begin{itemize}
    \item child:: -> pode ser omitido. child é o default axis.
    \item //e -> descendant-or-self::e
    \item ./e -> self::e
    \item ../e -> parent::e
    \item @e -> attribute::e
\end{itemize}

\paragraph{}

A função document.evaluate() pode ser usada para selecionar elementos usando expressões XPath.
Permite selecionar elementos não selecionáveis com alguns seletores de CSS.

\end{document}

