\documentclass[../resumosLTW.tex]{subfiles}

\newenvironment{conditions}
  {\par\vspace{\abovedisplayskip}\noindent\begin{tabular}{>{$}l<{$} @{${}={}$} l}}
  {\end{tabular}\par\vspace{\belowdisplayskip}}

\begin{document} 

Define um conjunto de regras para codificar documentos num formato que é legível por humanos e máquinas.

\paragraph{}

É uma metalinguagem permitindo que qualquer pessoa crie a sua própria linguagem para diferentes tipos de documentos.

\paragraph{}

Um documento XML é considerado bem formado se:
\begin{itemize}
    \item contém 1 ou mais elementos.
    \item tem uma e uma só raíz.
    \item os seus elementos se aninham adequadamente uns com os outros.
\end{itemize}

\paragraph{}

Desde o XML1.1 todos os documentos têm que começar com uma instrução que indica a sua versão, senão é considerado XML1.0.
A codificação por defeito é UTF-8.
\begin{lstlisting}
    <?xml version="1.1" encoding="utf-8"?>
\end{lstlisting}

\paragraph{}

Secções CDATA são usadas para incluir texto que inclui blocos que, noutro caso, seriam lidos como markup.
Começam com \lstinline{<![CDATA} e terminam com \lstinline{]]>}. Exemplo:
\begin{lstlisting}
    <![CDATA <warning>Not markup!</warning>]]>
\end{lstlisting}

\paragraph{}

Elementos são definidos com tag de abertura e fecho.
Todos têm que ser fechados.
Todos os elementos abertos dentro de outro elemento têm que ser fechados antes do pai.

\paragraph{}

Atributos são usados para associar pares nome-valor a um elemento. 
Só aparecem na tag inicial.

\end{document}

