\documentclass[../resumosLTW.tex]{subfiles}

\newenvironment{conditions}
  {\par\vspace{\abovedisplayskip}\noindent\begin{tabular}{>{$}l<{$} @{${}={}$} l}}
  {\end{tabular}\par\vspace{\belowdisplayskip}}

\begin{document} 

Código PHP tem que ser delimitado através de "\lstinline{<?php}" e "\lstinline{?>}", ou "\lstinline{<?}" e "\lstinline{?>}", ou "\lstinline{<script language="php">}" e "\lstinline{</script>}".

\paragraph{}

Imprimir strings:
\begin{itemize}
    \item \lstinline{<?php echo 'string'; ?>}
    \item \lstinline{<?= 'string' ?>}
\end{itemize}

\paragraph{}

Variáveis são representadas com \$ seguido do nome da variável.

\paragraph{}

A atribuição a uma variável é feita por valor exceto se o sinal \& for usado.

\paragraph{}

O valor null representa uma variável sem valor.

\paragraph{}

Uma variável é nula (null) se:
\begin{itemize}
    \item lhe foi atribuída a constante null.
    \item ainda não lhe foi atribuído um valor.
    \item foi utilizada a função \lstinline{unset()}.
\end{itemize}

\paragraph{}

As funções die e exit param a execução de um script PHP. Podem receber um argumento (status).

\paragraph{}

Dois tipos de comparação:
\begin{itemize}
    \item \lstinline{==} e \lstinline{!=}: comparam-se valores depois de haver cast para o mesmo tipo de variável.
    \item \lstinline{===} e \lstinline{!==}: comparam-se valores verdadeiros.
\end{itemize}

\paragraph{}

Strings podem ser definidas com aspas ou plicas. Strings com aspas expandem as variáveis que se encontram dentro.

\paragraph{}

Concatenação de strings é feita com o operador \textbf{\lstinline{.}}

\paragraph{}

Arrays:
\begin{itemize}
    \item Podem ser criados com a função array() ou [].
    \item São mapas organizados como coleção de pares key-value.
    \item Por defeito, as chaves são inteiros. Quando uma chave não é especificada, o PHP usa o incremento da maior chave inteira utilizada até ao momento.
\end{itemize}

\paragraph{}

Para utilizar uma variável global dentro de uma função é necessário declará-la como global.

\paragraph{}

Classes:
\begin{itemize}
    \item Declaradas com class.
    \item Podem ter atributos/métodos privados, protegidos ou públicos.
    \item Para utilizar um atributo num méto método é utilizada a keyword \$this.
    \item Podem ser declarados constructor (\_\_contstruct()) e destrutor (\_\_destruct()).
    \item Uma instância da classe é criada com a keyword new.
    \item Existe herança entre classes (extends) e é possível implementar interfaces (implements).
\end{itemize}

\paragraph{}

Para conectar a uma base de dados é utilizado um objeto PDO.

\paragraph{}

Para evitar ataques:
\begin{lstlisting}
    $stmt = $db->prepare(query);
    $stmt->execute(array(args));
    $stmt->fetch(); // Retira um resultado (fetchAll() retorna todos)
\end{lstlisting}

\paragraph{}

\$\_GET[] e \$\_POST[] permitem receber argumentos passados por estes métodos.

\paragraph{}

\$\_COOKIE permite aceder aos cookies enviados pelo browser.

\paragraph{}

Uma sessão é começada com a função \lstinline{session_start()}. A variável \$\_SESSION guarda um array com as informações importantes da sessão. A função \lstinline{session_destroy()} destrói todos os dados associados à sessão atual.



\end{document}

