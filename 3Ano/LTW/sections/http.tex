\documentclass[../resumosLTW.tex]{subfiles}

\newenvironment{conditions}
  {\par\vspace{\abovedisplayskip}\noindent\begin{tabular}{>{$}l<{$} @{${}={}$} l}}
  {\end{tabular}\par\vspace{\belowdisplayskip}}

\begin{document} 

URI: Uniform Resource Identifier
\begin{itemize}
    \item Um identificador é um objeto que aje como referência para algo que tem identidade.
    \item Um URI pode ser classificado como localizador (URL), nome (URN) ou ambos.
    \item Componentes de um URI: esquema, autoridade, path, query, fragmento.
\end{itemize}

\paragraph{}

URN: Uniform Resource Names
\begin{itemize}
    \item Destinam-se a servir como identificadores de recursos persistentes e independentes do local.
\end{itemize}

\paragraph{}

URL: Uniform Resource Locator
\begin{itemize}
    \item Refere-se ao subconjunto de URI que identifica recursos através de uma representação do seu mecanismo de acesso primário, em vez de identificar o recurso por nome ou outro atributo desse recurso.
    \item \lstinline{scheme://domain:port/path?query_string#fragment_id}
    \begin{itemize}
        \item scheme refere-se ao protocolo.
        \item port, query\_string e fragment\_id são opcionais. 
    \end{itemize}
\end{itemize}

\paragraph{}

Uma sessão HTTP consiste em 3 fases:
\begin{itemize}
    \item O cliente estabelece uma conexão TCP.
    \item O cliente envia um pedido e espera por uma resposta.
    \item O servidor processa o pedido e envia a resposta, contendo um status code e os dados apropriados.
\end{itemize}

\paragraph{}

Pedido HTTP: a primeira linha contém o método seguido dos seus parâmetros.
O método indica a ação a realizar.

\paragraph{}


Um método seguro não tem efeitos secundários no servidor.
\begin{itemize}
    \item GET: usado para recuperar informação identificada pelo URI pedido.
    \item HEAD: idêntico ao GET, mas sem o envio do corpo da mensagem.
\end{itemize}

\paragraph{}

Um método idempotente é um método onde os efeitos secundários de vários pedidos idênticos são os mesmos efeitos secundários de um pedido.
\begin{itemize}
    \item PUT: solicita que a entidade fechada seja armazenada no URI fornecido.
    \item DELETE: apaga o recurso identificado pelo URI.
\end{itemize}

\paragraph{}

Outros métodos:
\begin{itemize}
    \item POST: solicita que o servidor aceite a entidade fechada no pedido como um novo subordinado do recurso identificado pelo URI.
    \item OPTIONS, TRACE, CONNECT e PATH.
\end{itemize}

\paragraph{}

Ao responder a um pedido de um cliente, o servidor envia 1 código e 3 dígitos:
\begin{itemize}
    \item de informação (1XX).
    \item de sucesso (2XX).
    \item de redirecionamento (3XX).
    \item de erro do cliente (4XX).
    \item de erro do servidor (5XX).
\end{itemize}

\paragraph{}

Para descobrir qual método HTTP foi usado para aceder a um recurso usa-se o array \$\_SERVER.
\begin{itemize}
    \item \$\_SERVER['REQUEST\_METHOD']
\end{itemize}

\end{document}

