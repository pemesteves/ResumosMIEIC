\documentclass[../resumosLTW.tex]{subfiles}

\newenvironment{conditions}
  {\par\vspace{\abovedisplayskip}\noindent\begin{tabular}{>{$}l<{$} @{${}={}$} l}}
  {\end{tabular}\par\vspace{\belowdisplayskip}}

\begin{document} 

Alguns elementos HTML podem ter filhos. Estes elementos têm tag de abertura e fecho. Elementos que não permitem ter filhos só têm tag e abertura.

\paragraph{}

Um documento HTML tem uma tag html como raíz e as secções head (que tem que conter a tag title) e body. A tag html deve conter o atributo lang.

\paragraph{}

Listas:
\begin{itemize}
    \item Ordenadas (ol).
    \item Não ordenadas (ul).
    \item Descritivas (dl): definem-se termos (dt) e as suas descrições (dd).
\end{itemize}

\paragraph{}

Tabelas (table):
\begin{itemize}
    \item Pode ter um título/legenda (caption).
    \item Organizada através de linhas (tr) que contêm células de dados (td).
    \item Algumas células podem ser cabeçalhos (headers): th em vez de td.
    \item colspan: célula ocupa número de colunas especificado.
    \item rowspan: célula ocupa número de linhas especificado.
    \item Secções: thead, tfoot, tbody: permite especificar header, footer e body de uma tabela.
    \item Grupos de colunas: para não ser repetida informação nas colunas, definem-se grupos de colunas com as tags colgroup e col.
\end{itemize}

\paragraph{}

Formulários (form):
\begin{itemize}
    \item Dois atributos importantes
    \begin{itemize}
        \item action: página para onde serão enviados os resultados.
        \item method: get ou post.
    \end{itemize}
    \item Controlos:
    \begin{itemize}
        \item input: campos editáveis.
        \item textarea: campo de texto editável.
        \item select: dropdown list.
        \item button: botão genérico.
    \end{itemize}
    \item Select (dropdown list):
    \begin{itemize}
        \item Cada opção é representada com a tag option.
        \item Opções podem ser agrupadas com optgroup (atributo label indica o nome do grupo).
    \end{itemize}
    \item Label: permite associação entre uma label e o respetivo input. Clicando na label o input é ativado.
    \item Field Set (fieldset): permite agrupar inputs. Tag legend contém o título do grupo.
\end{itemize}

\paragraph{}

Um documento HTML não pode ter dois elementos com o mesmo id.

\paragraph{}

Elementos semânticos (alternativa a div): header, nav, aside, section, article, footer.

\end{document}

