\documentclass[../resumosLTW.tex]{subfiles}

\newenvironment{conditions}
  {\par\vspace{\abovedisplayskip}\noindent\begin{tabular}{>{$}l<{$} @{${}={}$} l}}
  {\end{tabular}\par\vspace{\belowdisplayskip}}

\begin{document} 

Seletores: permitem selecionar elementos HTML a serem alterados.

\paragraph{}

Propriedades: aspetos a alterar nos elementos selecionados.

\paragraph{}

\lstinline{<link rel="stylesheet" href="style.css">}: ligar ficheiro HTML ao ficheiro \emph{style.css}.

\paragraph{}

\lstinline{>}: seleciona "filhos".

\paragraph{}

\lstinline{+}: seleciona próximo "irmão".

\paragraph{}

\lstinline{~}: seleciona próximos "irmãos". 

\paragraph{}

Pseudo-classes permitem selecionar elementos com determinadas características.

\paragraph{}

Seletores de atributos: seleciona elementos baseando-se na existência de atributos e valores (exemplo: \lstinline{form[method=get]}.

\paragraph{}

Tamanhos relativos de fontes:
\begin{itemize}
    \item \textbf{em}: quando usado com font-size, representa o tamanho da fonte do elemento-"pai". Para comprimentos, representa o tamanho da fonte do elemento atual.
\end{itemize}

\paragraph{}

Flexbox: alternativa ao box layout model. Elementos são flexíveis e adaptam-se ao tamanho atual do ecrã. É necessário alterar o atributo display para flex.

\paragraph{}

Grid: permite alinhar elementos em linhas e colunas.

\paragraph{}

Cálculo de especificidade: 
\begin{itemize}
    \item A especificidade de uma regra é definida como 4 valores (a, b, c, d).
    \item Cada valor é incrementado quando um certo tipo de seletor é utilizado:
    \begin{itemize}
        \item d: Elemento, Pseudo-elemento (::).
        \item c: Classe, Pseudo-classe (:), Atributo.
        \item b: id.
        \item a: Inline style.
    \end{itemize}
\end{itemize}

\paragraph{}

CSS Vars:
\begin{itemize}
    \item Entidades que podem ser reutilizadas no documento.
    \item Acedem-se com a função \lstinline{var()}.
\end{itemize}

\end{document}

