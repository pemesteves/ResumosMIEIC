\documentclass[../resumosLTW.tex]{subfiles}

\newenvironment{conditions}
  {\par\vspace{\abovedisplayskip}\noindent\begin{tabular}{>{$}l<{$} @{${}={}$} l}}
  {\end{tabular}\par\vspace{\belowdisplayskip}}

\begin{document} 

Para usar ECMAScript 5, o ficheiro de JavaScript deve começar com 'use strict'.
\begin{itemize}
    \item Alterações:
    \begin{itemize}
        \item Não existem variáveis globais não declaradas.
        \item Não se declaram variáveis com var.
        \item Alguns warnings passaram a erros.
    \end{itemize}
\end{itemize}

\paragraph{}

Variáveis são declaradas com let. Os seus nomes podem ter apenas números, letras, \$ e \_ (e não podem começar com um número).

\paragraph{}

Constantes são declaradas com const.

\paragraph{}

Variáveis declaradas com var:
\begin{itemize}
    \item Não têm block scope (só function scope).
    \item São processadas quando uma função começa.
\end{itemize}

\paragraph{}

Tipos de dados primitivos:
\begin{itemize}
    \item Number (double).
    \item String (text).
    \item Boolean (true ou false).
    \item Null (apenas 1 valor: null).
    \item Undefined (ainda não foi atribuído um valor).
\end{itemize}

\paragraph{}

Strings podem ser definidas com plicas, aspas ou backticks (`). Com a última, expressões dentro de \$\{...\} são avaliadas e o resultado faz parte da string.

\paragraph{}

Operador + soma números ou concatena strings (se pelo menos um operando for uma string).

\paragraph{}

É possível usar String(), Boolean() e Number() para converter valores para os tipos pretendidos.

\paragraph{}

\lstinline{===} e \lstinline{!==} comparam valores. 
\lstinline{==} e \lstinline{!=} comparam tipos.

\paragraph{}

Funções são definidas com a keyword function. Tipos primitivos são passados por valor, mas tipos não primitivos são passados por referência. Funções com um return vazio ou sem return, retornam undefined.

\paragraph{}

Arrow functions são uma forma mais compacta de declarar funções.

\paragraph{}

Pode-se usar a keyword this dentro de um objeto para referir ao objeto atual.

\paragraph{}

Call e Apply são alternativas a chamadas a funções. Ambas recebem o contexto como primeiro argumento.

\paragraph{}

Objetos podem ser acedidos com [] como num array associativo.

\paragraph{}

\lstinline{for ... in} permite executar código para cada propriedade de um objeto/array.

\paragraph{}

Cada função tem uma propriedade prototype interna que é inicializada como um objeto vazio.
Quando o new é utilizado, é criado um novo objeto derivado do prototype.
É possível alterar o prototype de uma função deretamente.

\paragraph{}

Quando um objeto é criado com new, uma propriedade \_\_proto\_\_ é inicializada com o prototype da função que o criou.

\paragraph{}

A keyword class é uma forma simplificada de criar classes protoype-based.
Só podem ter métodos e getters/setters.

\paragraph{}

Arrays são objetos list-like.
Só podem ser acedidos com [].
Os seus índices começam em 0.

\paragraph{}

Podem ser lançadas exceções (ou qualquer outro objeto) com throw.

\paragraph{}

Exceções devem ser envolvidas em blocos \lstinline{try .. catch}.

\paragraph{}

JavaScript pode ser incluído num ficheiro HTML com a tag script.

\paragraph{}

Scripts devem ter 1 dos seguintes métodos:
\begin{itemize}
    \item async: é corrido assim que seja feito o seu download sem bloquear o browser.
    \item defer: é corrido apenas quando a página é carregada e por ordem.
\end{itemize}

\paragraph{}

document representa o documento atual.
Contém métodos importantes para aceder a elementos.
Um objeto Element representa um elemento HTML (id, innerHTML, outerHTML, style, getAttribute(), setAttribute(), remove()).
A função createElement() de document cria um novo elemento, mas não é inserido no documento.
O objeto Node representa um nó na árvore do documento (appendChild(), replaceChild(), removeChild(), insertBefore()).

\paragraph{}

XMLHttpRequest permite enviar pedidos HTTP facilmenteL
\begin{itemize}
    \item open(method, url, async)
    \begin{itemize}
        \item Method: get ou post.
        \item URL: URL a procurar.
        \item Async: se for falso, a execução para à espera de uma resposta.
    \end{itemize}
    \item encodeForAjax(obj): cria um objeto a passar ao PHP.
    \item send(): envia os dados.
    \item JSON.parse(): permite receber a resposta do servidor.
\end{itemize}


\end{document}

