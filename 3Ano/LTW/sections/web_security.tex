\documentclass[../resumosLTW.tex]{subfiles}

\newenvironment{conditions}
  {\par\vspace{\abovedisplayskip}\noindent\begin{tabular}{>{$}l<{$} @{${}={}$} l}}
  {\end{tabular}\par\vspace{\belowdisplayskip}}

\begin{document} 

\subsection{Path Traversal Attack}

Utilização de .. e / para aceder a ficheiros e diretórios que não devem ser acedidos.

\subsection{SQL Injection}

Injeção e queries SQL a partir de dados de input.

\subsection{Account Lockout}

A aplicação contém bloqueio de contas que pode ser ativado facilmente (mecanismo usado contra ataques de força bruta).

Isto permite que os atacantes bloqueiem serviços aos utilizadores, bloqueando-lhes as contas.

\subsection{Cross-Site Scripting (XSS)}

Injeção de scripts malignos em sites.

\subsection{Cross-Site Request Forgery (CSRF)}

Um atacante cria links que levam para ações na página onde o utilizador já tem sessão iniciada.

\subsection{Man in the Middle Attack}

Intercetar comunicações entre dois sistemas.

Evita-se usando chaves públicas assinadas por uma certificate authority (CA).

\subsection{Credential Storage}

Passwords devem ser passadas ao servidor e só depois se faz uma hash, utilizando um salt variável.

Um salt deve ser gerado com um Cryptographically Secure Pseudo-Random Number Generator. 
Deve ser único para cada user.
Tem que ser longo.
O salt deve ser concatenado à password e, de seguida, deve ser criada uma hash.
Tanto o salt como a hash devem ser guardados na base de dados.

\subsection{Session Fixation}

Obter ID de sessão válido, induzindo um utilizador a autenticar-se com esse ID e, posteriormente, roubar a sessão validada com o conhecimento do ID de sessão utilizado.

\subsection{Session Hijacking}

Ganhar controlo da sessão do utilizador roubando o ID de sessão.

\subsection{Denial of Service}

Tentativa de fazer uma máquina ou recurso de rede indisponível para os seus utilizadores.

\end{document}

